\documentclass[10pt]{article}
\usepackage[T1]{fontenc}
\usepackage[headheight=1.5cm,top=3cm,left=2cm,right=2cm]{geometry}
\usepackage[scaled=0.8]{beramono}
\usepackage{pdflscape}
\usepackage{amsmath}
\usepackage{amsfonts}
\usepackage{helvet}
\usepackage{longtable}
\usepackage{hyperref}
\usepackage{graphicx}

\usepackage{caption}
\usepackage{subcaption}

\graphicspath{{images/}}
\usepackage[dvipsnames]{xcolor}
\usepackage{float}
\usepackage{fancyhdr}
\usepackage{multirow}
\usepackage{array}
\newcolumntype{L}[1]{>{\raggedright\let\newline\\\arraybackslash\hspace{0pt}}m{#1}}
\newcolumntype{C}[1]{>{\centering\let\newline\\\arraybackslash\hspace{0pt}}m{#1}}
\newcolumntype{R}[1]{>{\raggedleft\let\newline\\\arraybackslash\hspace{0pt}}m{#1}}

\usepackage{stackengine}
\newcommand\No[1][.13ex]{%
  \setbox0=\hbox{\scalebox{.7}{o}}%
  \setbox2=\hbox{N}%
  N\kern-.05em\stackengine{\dimexpr\ht0-\ht2+#1}{\belowbaseline[-\ht2]{\copy0}}%
    {\rule[-.13ex]{.7\wd0}{.13ex}}%
    {U}{c}{F}{F}{L}%
}

\usepackage{pbox}
%\usepackage{tikz, pgfplots}
\usepackage{pgf,tikz,pgfplots}
\pgfplotsset{compat=1.15}
\usepackage{mathrsfs}
\usetikzlibrary{arrows}
\definecolor{qqqqff}{rgb}{0,0,1}
\definecolor{ffqqqq}{rgb}{1,0,0}
\usepackage{color, colortbl}
\usetikzlibrary{arrows,positioning}
\renewcommand{\familydefault}{\sfdefault}
\setlength{\parindent}{0pt}
\setlength{\parskip}{10pt}
\pagestyle{fancy}
\renewcommand{\headrulewidth}{0.0pt}
\fancyhf{}
\usepackage{lastpage}

\newcommand{\doctitle}{Semantic Segmentation of Galilei Colour TopView images}
\newcommand{\docauthor}{Fabian M\"uller}
\newcommand{\docnr}{n/a}
\newcommand{\docversion}{n/a}
\newcommand{\doctempl}{n/a}

\title{\doctitle}
\author{\docauthor}

\renewcommand{\labelenumii}{\theenumii}
\renewcommand{\theenumii}{\theenumi.\arabic{enumii}.}
\usepackage{enumitem}

\usepackage{listings}
%#%% HEDAER FOR KERAS REPORTS %%%
\usepackage[T1]{fontenc}
\usepackage[headheight=1.5cm,top=3cm,left=2cm,right=2cm]{geometry}
\usepackage{amsfonts}
\usepackage{helvet}
\usepackage{longtable}
\usepackage{hyperref}
\usepackage{graphicx}

\usepackage{caption}
\usepackage{subcaption}

\usepackage[dvipsnames]{xcolor}
\usepackage{float}
\usepackage{fancyhdr}
\usepackage{multirow}
\usepackage{array}
\newcolumntype{L}[1]{>{\raggedright\let\newline\\\arraybackslash\hspace{0pt}}m{#1}}
\newcolumntype{C}[1]{>{\centering\let\newline\\\arraybackslash\hspace{0pt}}m{#1}}
\newcolumntype{R}[1]{>{\raggedleft\let\newline\\\arraybackslash\hspace{0pt}}m{#1}}

%# For the No. sign
\usepackage{stackengine}
\newcommand\No[1][.13ex]{%
  \setbox0=\hbox{\scalebox{.7}{o}}%
  \setbox2=\hbox{N}%
  N\kern-.05em\stackengine{\dimexpr\ht0-\ht2+#1}{\belowbaseline[-\ht2]{\copy0}}%
    {\rule[-.13ex]{.7\wd0}{.13ex}}%
    {U}{c}{F}{F}{L}%
}

%# for plots
\usepackage{pbox}
\usepackage{pgf,tikz,pgfplots}
\pgfplotsset{compat=1.15}
\usepackage{mathrsfs}
\usetikzlibrary{arrows}
\definecolor{qqqqff}{rgb}{0,0,1}
\definecolor{ffqqqq}{rgb}{1,0,0}

\usepackage{color, colortbl}
\usetikzlibrary{arrows,positioning}
\renewcommand{\familydefault}{\sfdefault}
\setlength{\parindent}{0pt}
\setlength{\parskip}{10pt}
\pagestyle{fancy}
\renewcommand{\headrulewidth}{0.0pt}
\fancyhf{}


\renewcommand{\labelenumii}{\theenumii}
\renewcommand{\theenumii}{\theenumi.\arabic{enumii}.}
\usepackage{enumitem}

\usepackage{listings}

\newcommand{\dev}[1]{{\sc #1}}

\usepackage{siunitx}
\usepackage{tabularx}
\newcolumntype{L}[1]{>{\raggedright\arraybackslash}p{#1}}
\newcolumntype{C}[1]{>{\centering\arraybackslash}p{#1}}
\newcolumntype{R}[1]{>{\raggedleft\arraybackslash}p{#1}}


\newcommand{\dev}[1]{{\sc #1}}

%#\usepackage{showkeys}
\usepackage{longtable}
\usepackage{siunitx}
\usepackage{tabularx}
\newcolumntype{L}[1]{>{\raggedright\arraybackslash}p{#1}} % linksbündig mit Breitenangabe
\newcolumntype{C}[1]{>{\centering\arraybackslash}p{#1}} % zentriert mit Breitenangabe
\newcolumntype{R}[1]{>{\raggedleft\arraybackslash}p{#1}} % rechtsbündig mit Breitenangabe

% %%% COLOURS FOR LAYERS
% GREEN 81,168,50
% RED Conv 168,50,95
% BLUE Maxpool 50,58,168
% PURPLE upsampling 87,50,168
% RED Concatenate 168,58,50
% Orange Activation 235,128,52
% Turquoise BatchNorm 52,235,174
% PURPLE Additive 174,52,235
\definecolor{InputLayer}{RGB}{52, 220, 235} %light blue

\definecolor{Conv2D}{RGB}{235, 101, 52} % orange/red
\definecolor{Dense}{RGB}{235, 52, 55} %red
\definecolor{BatchNormalization}{RGB}{235, 52, 140} %pink

\definecolor{Activation}{RGB}{92, 52, 235} %blue
\definecolor{MaxPooling2D}{RGB}{174, 52, 235} %purple
\definecolor{UpSampling2D}{RGB}{52, 107, 235} %lighter blue

\definecolor{Concatenate}{RGB}{81,168,50} %Green
\definecolor{Add}{RGB}{50, 168, 121} % Blueish green
% %%%%%%%%%%%%%%%%%%%%%%
\setcounter{tocdepth}{2}
\begin{document}

%# TITLE
$\,$\\[-2ex]
\begin{flushright}
    {\huge{\bf
    \doctitle
    }}\\[1ex]
    {\large
    \docauthor
    }
\end{flushright}
%\frontpages{Function Author}{Author}{Function Reviewer}{Reviewer}{Function Release}{Release}
\tableofcontents
% %%%%%%%%%%%%%%%%%%%%%%%%%%%%%%%%%%%%%%%%%%%%%%%%%%%%%%%%%%%%%%%%%%%
\section{Summary}
\begin{tabular}{rrrrrrrr}
    \hline\\[-1.5ex]
    \No{} & Model name & Pretrained & \#Parameters & \#Epochs & Batch size & Test Acc. & Training Acc. \\
    \hline\\[-1.5ex]

    %% for summary in summaries
    \hyperref[training:\VAR{summary['model']['reportnumber']}]
             {\VAR{summary['model']['reportnumber']}} &
    \hyperref[model:\VAR{summary['model']['name']}]
             {\VAR{summary['model']['name']}} &
    \VAR{summary['model']['is_pretrained']} &
    \num{\VAR{summary['model']['n_params']}} &
    \VAR{summary['training']['epochs']}
    %% if summary['model']['pretrained_epochs'] is defined
    + \VAR{summary['model']['pretrained_epochs']}
    %% endif
    &
    \VAR{summary['training']['batch_size']} &
    \VAR{summary['training']['history']['val_categorical_accuracy_perc']} \% &
    \VAR{summary['training']['history']['categorical_accuracy_perc']} \%
    \\[4pt]
    %% endfor
    \hline
\end{tabular}
% %%%%%%%%%%%%%%%%%%%%%%%%%%%%%%%%%%%%%%%%%%%%%%%%%%%%%%%%%%%%%%%%%%%
\newpage
% %%%%%%%%%%%%%%%%%%%%%%%%%%%%%%%%%%%%%%%%%%%%%%%%%%%%%%%%%%%%%%%%%%%
\section{Training reports}
%% for summary in summaries
    \subsection{Model \VAR{summary['model']['reportnumber']}: \VAR{summary['model']['name']}
                \label{training:\VAR{summary['model']['reportnumber']}}}
    %
    \paragraph*{Training history} See Figure \ref{fig:results\VAR{summary['model']['reportnumber']}}.
    \begin{figure}[H]
        \centering
        \begin{subfigure}{.5\textwidth}
            \VAR{summary['plots_tikz']['acc']}
            \caption{Accuracy learning process for model \protect\hyperref[training:\VAR{summary['model']['reportnumber']}]
                        {\VAR{summary['model']['reportnumber']}}.}
        \end{subfigure}%
        \hfill%
        \begin{subfigure}{.5\textwidth}
            \VAR{summary['plots_tikz']['loss']}
            \caption{Loss learning process for model \protect\hyperref[training:\VAR{summary['model']['reportnumber']}]
                        {\VAR{summary['model']['reportnumber']}}.}
        \end{subfigure}
        \par\bigskip
        \begin{subfigure}{.5\textwidth}
            \VAR{summary['plots_tikz']['lr']}
            \caption{Learning rate per epoch for model \protect\hyperref[training:\VAR{summary['model']['reportnumber']}]
                        {\VAR{summary['model']['reportnumber']}}.}
        \end{subfigure}%
        \hfill%
        \begin{subfigure}{.5\textwidth}
            \VAR{summary['plots_tikz']['eval']}
            \caption{Histogram Test set accuracies for model
            \protect\hyperref[training:\VAR{summary['model']['reportnumber']}]
                        {\VAR{summary['model']['reportnumber']}}.}
        \end{subfigure}
        \caption{Training and evaluation metrics for model  \protect\hyperref[training:\VAR{summary['model']['reportnumber']}]
                    {\VAR{summary['model']['reportnumber']}}.
                \label{fig:results\VAR{summary['model']['reportnumber']}}}
    \end{figure}

    \subsubsection*{Dataset}
    \begin{description}
        \item[Train-Test-Dev split:] {\it Training set:}
        \VAR{summary['data']['samples']['train']['n_samples']}
        (\VAR{summary['data']['samples']['train']['ratio_perc']} \%),
        {\it Test set:}
        \VAR{summary['data']['samples']['test']['n_samples']}
        (\VAR{summary['data']['samples']['test']['ratio_perc']} \%),
        {\it Dev set:}
        \VAR{summary['data']['samples']['dev']['n_samples']}
        (\VAR{summary['data']['samples']['dev']['ratio_perc']} \%)
        \item[Image size] \VAR{summary['data']['image_size']}
        \item[Augmentations]
            \begin{enumerate}
                \item Horizontal flip,
                \item Vertical flip,
                \item Rotation around centre by $\VAR{summary['data']['aug_rotangle']}^{\circ}$,
                \item Random noise of type: \VAR{summary['data']['aug_noisetype']}.
            \end{enumerate}
        \item[Path to .pickle file] \VAR{summary['data']['path_pickle']}
    \end{description}
    %
    \subsubsection*{Training}
    \begin{description}
        \item[Number of epochs] \VAR{summary['training']['epochs']}
        \item[Optimiser] \VAR{summary['training']['optimizer'].capitalize()}
        \item[Learning rate] \VAR{summary['training']['history']['lr'][0]}
        \item[Loss] \VAR{summary['training']['loss'].capitalize()}
        \item[Batch size] \VAR{summary['training']['batch_size']}
        \item[Shuffle] \VAR{summary['training']['shuffle']}
        \item[Training time] \VAR{summary['metadata']['training_time']}
    \end{description}
    %
    \subsubsection*{Miscellaneous}
    \begin{description}
        \item[LED removal] {\it Region:} $x$ in \VAR{summary['data']['led_region'][:2]},
                                         $y$ in \VAR{summary['data']['led_region'][-2:]}.
                           {\it Threshold:} \VAR{summary['data']['led_threshold']}.
                           {\it Replacement value:} \VAR{summary['data']['led_replace']}.
    \end{description}
    %
    \subsubsection*{Platform}
    \begin{description}
        \item[Weights exported to path] \VAR{summary['model']['weights_path']}
        \item[Device used] \VAR{summary['metadata']['training_device']['device_string']}
        \item[CPU] \VAR{summary['metadata']['training_device']['cpu']['brand']},
                   \VAR{summary['metadata']['training_device']['cpu']['arch']}
        \item[Python Version] \VAR{summary['metadata']['training_device']['python_version']}
        \item[Keras Version] \VAR{summary['metadata']['keras_version']} (Backend: \VAR{summary['metadata']['keras_backend']})
        \item[Tensorflow Version] \VAR{summary['metadata']['tensorflow_version']}
        \item[Timestamp] \VAR{summary['metadata']['starttime']}
    \end{description}
%% endfor
% %%%%%%%%%%%%%%%%%%%%%%%%%%%%%%%%%%%%%%%%%%%%%%%%%%%%%%%%%%%%%%%%%%%
\newpage
% %%%%%%%%%%%%%%%%%%%%%%%%%%%%%%%%%%%%%%%%%%%%%%%%%%%%%%%%%%%%%%%%%%%
\section{Model Architectures}
%% for model in model_list
\subsection{\VAR{model['modelname']}
            \label{model:\VAR{model['modelname']}}
            }
\paragraph{Used in \No{}:}
    %% for k in model['model_idc'][:-1]
    \hyperref[training:\VAR{summaries[k]['model']['reportnumber']}]
             {\VAR{summaries[k]['model']['reportnumber']}},
    %% endfor
    \hyperref[training:\VAR{summaries[-1]['model']['reportnumber']}]
             {\VAR{summaries[-1]['model']['reportnumber']}}
\vspace{-2ex}
%
\paragraph{Model summary:}$\,$\\
\begin{longtable}{rR{.22\textwidth}cL{.22\textwidth}rL{.22\textwidth}}
    \hline\\[-1.5ex]
    \No{} &
    Layer (Type) &
    Output shape &
    Config &
    \#Parameters &
    Inbound layers\\
    \hline\\
    \endhead
    %% for layer in summaries[model['model_idc'][0]]['model']['layers']
        \VAR{layer['number']}
        &
        \color{\VAR{layer['layertype']}}{
        \VAR{layer['name']}}
        \color{\VAR{layer['layertype']}}{
        (\VAR{layer['layertype']})}
        &
        \VAR{layer['output_shape']}
        &
        %% if layer['layertype'] == 'Conv2D'
            {\bf Activation:} \VAR{layer['activation']} \newline
            {\bf Kernel Size:} \VAR{layer['kernel_size']} \newline
            {\bf Stride:} \VAR{layer['strides']} \newline
            {\bf Dilation:} \VAR{layer['dilation_rate']} \newline
            {\bf Padding:} \VAR{layer['padding']}
        %% elif layer['layertype'] == 'MaxPooling2D'
            {\bf Pool size:} \VAR{layer['pool_size']} \newline
            {\bf Strides:} \VAR{layer['strides']} \newline
            {\bf Padding:} \VAR{layer['padding']}
        %% elif layer['layertype'] == 'UpSampling2D'
            {\bf Interpolation:} \VAR{layer['interpolation']}  \newline
            {\bf Scale factors:} \VAR{layer['scale_factors']}
        %% elif layer['layertype'] == 'Concatenate'
            {\bf Axis:} \VAR{layer['concat_axis']}
        %% elif layer['layertype'] == 'BatchNormalization'
            {\bf Momentum:} \VAR{layer['momentum']} \newline
            {\bf Additive Offset:} \VAR{layer['center']} \newline
            {\bf Scaling:} \VAR{layer['scale']} \newline
            {\bf Axis:} \VAR{layer['axis']}
        %% elif layer['layertype'] == 'Activation'
            {\bf Activation:} \VAR{layer['activation']}
        %% elif layer['layertype'] == 'Activation'
            {\bf Activation:} \VAR{layer['activation']}
        %% elif layer['layertype'] == 'InputLayer'
        %% elif layer['layertype'] == 'Add'
        %% else
            Parameters of layers of type \VAR{layer['layertype']} not implemented.
        %% endif
        &
        \num{\VAR{layer['n_params']}}
        &
        %% for in_layer in layer['inbound_layers'][:-1]
            \color{\VAR{in_layer['layertype']}}{
            \VAR{in_layer['name']}},
        %% endfor
        %% if layer['inbound_layers']|length
        \color{\VAR{layer['inbound_layers'][-1]['layertype']}}{
            \VAR{layer['inbound_layers'][-1]['name']}}
        %%endif
        \\
        \hline\\[-1.5ex]
    %% endfor
\end{longtable}
\newpage
%% endfor
% %%%%%%%%%%%%%%%%%%%%%%%%%%%%%%%%%%%%%%%%%%%%%%%%%%%%%%%%%%%%%%%%%%%

\end{document}
